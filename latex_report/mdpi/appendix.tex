% Appendix - BDCC Format
% Supplementary tables and ablation details (transformer only)

\section{Detailed Experimental Results}
\label{app:detailed}

This appendix provides comprehensive metrics and supplementary ablation studies for all experiments reported in the main text.

%-------------------------------------------------------------------
\subsection{Complete Metrics: Attention Ablation}
\label{app:se_tap_full}

Table~\ref{tab:se_tap_full} presents complete training, validation, and test metrics for the attention mechanism ablation study (Section~\ref{subsec:baseline}).

\begin{table}[H]
\caption{Complete metrics for SE/TAP ablation. All models use single-stream Kalman architecture with embed\_dim=64, 21-fold LOSO-CV.}
\label{tab:se_tap_full}
\centering
\small
\begin{tabular}{lcccccccc}
\toprule
& \multicolumn{2}{c}{\textbf{Train}} & \multicolumn{3}{c}{\textbf{Validation}} & \multicolumn{3}{c}{\textbf{Test}} \\
\cmidrule(lr){2-3} \cmidrule(lr){4-6} \cmidrule(lr){7-9}
\textbf{Config} & \textbf{F1} & \textbf{Loss} & \textbf{F1} & \textbf{Acc} & \textbf{Loss} & \textbf{F1} & \textbf{Acc} & \textbf{Loss} \\
\midrule
No SE/TAP & 90.21 & 0.00043 & 94.21 & 91.42 & 0.00035 & 88.52 & 83.67 & 0.00131 \\
SE only & 91.45 & 0.00038 & 94.89 & 91.89 & 0.00032 & 89.15 & 84.23 & 0.00107 \\
TAP only & 90.12 & 0.00042 & 93.95 & 91.18 & 0.00036 & 88.34 & 83.41 & 0.00108 \\
\textbf{SE+TAP} & \textbf{92.06} & \textbf{0.00034} & 95.97 & 92.02 & 0.00033 & \textbf{89.80} & \textbf{84.96} & \textbf{0.00083} \\
\bottomrule
\end{tabular}
\end{table}

%-------------------------------------------------------------------
\subsection{Complete Metrics: Embedding Ratio Ablation}
\label{app:embed_full}

Tables~\ref{tab:embed_kalman_full} and~\ref{tab:embed_raw_full} present complete metrics for embedding ratio experiments with Kalman-fused and raw gyroscope inputs respectively.

\begin{table}[H]
\caption{Complete metrics for dual-stream embedding ratio ablation with Kalman-fused input.}
\label{tab:embed_kalman_full}
\centering
\small
\begin{tabular}{lccccccc}
\toprule
& & \multicolumn{2}{c}{\textbf{Train}} & \multicolumn{2}{c}{\textbf{Val}} & \multicolumn{2}{c}{\textbf{Test}} \\
\cmidrule(lr){3-4} \cmidrule(lr){5-6} \cmidrule(lr){7-8}
\textbf{Split} & \textbf{Dim} & \textbf{F1} & \textbf{Loss} & \textbf{F1} & \textbf{Loss} & \textbf{F1} & \textbf{Loss} \\
\midrule
32:32 & 64 & 92.06 & 0.00034 & 96.10 & 0.00033 & \textbf{91.10} & 0.00083 \\
48:48 & 96 & 92.79 & 0.00032 & 95.81 & 0.00027 & 89.58 & 0.00587 \\
48:24 & 72 & 92.56 & 0.00033 & 96.08 & 0.00028 & 89.05 & 0.00568 \\
\bottomrule
\end{tabular}
\end{table}

\begin{table}[H]
\caption{Complete metrics for dual-stream embedding ratio ablation with raw gyroscope input.}
\label{tab:embed_raw_full}
\centering
\small
\begin{tabular}{lccccccc}
\toprule
& & \multicolumn{2}{c}{\textbf{Train}} & \multicolumn{2}{c}{\textbf{Val}} & \multicolumn{2}{c}{\textbf{Test}} \\
\cmidrule(lr){3-4} \cmidrule(lr){5-6} \cmidrule(lr){7-8}
\textbf{Split} & \textbf{Dim} & \textbf{F1} & \textbf{Loss} & \textbf{F1} & \textbf{Loss} & \textbf{F1} & \textbf{Loss} \\
\midrule
48:48 & 96 & 95.74 & 0.00022 & 92.47 & 0.00059 & 87.58 & 0.01259 \\
48:24 & 72 & 95.27 & 0.00023 & 92.54 & 0.00062 & 87.63 & 0.01140 \\
24:48 & 72 & 95.16 & 0.00024 & 92.33 & 0.00066 & 88.06 & 0.00977 \\
\bottomrule
\end{tabular}
\end{table}

%-------------------------------------------------------------------
\subsection{Complete Metrics: Single-Stream Baselines}
\label{app:single_full}

\begin{table}[H]
\caption{Complete metrics for single-stream architectures (embed\_dim=64).}
\label{tab:single_full}
\centering
\small
\begin{tabular}{lccccccc}
\toprule
& \multicolumn{2}{c}{\textbf{Train}} & \multicolumn{2}{c}{\textbf{Val}} & \multicolumn{2}{c}{\textbf{Test}} \\
\cmidrule(lr){2-3} \cmidrule(lr){4-5} \cmidrule(lr){6-7}
\textbf{Input} & \textbf{F1} & \textbf{Loss} & \textbf{F1} & \textbf{Loss} & \textbf{F1} & \textbf{Loss} \\
\midrule
Kalman & 92.28 & 0.00034 & 95.97 & 0.00025 & 89.80 & 0.00699 \\
Raw & 94.34 & 0.00028 & 92.83 & 0.00058 & 88.96 & 0.00994 \\
\bottomrule
\end{tabular}
\end{table}

%-------------------------------------------------------------------
\section{Supplementary Ablation Studies}
\label{app:supplementary}

%-------------------------------------------------------------------
\subsection{Kalman Filter Design}
\label{app:kalman_design}

Table~\ref{tab:kalman_ablation} presents ablation studies on Kalman filter design choices.

\begin{table}[H]
\caption{Kalman filter design ablation. All experiments use dual-stream transformer with SE+TAP.}
\label{tab:kalman_ablation}
\centering
\begin{tabular}{lcc}
\toprule
\textbf{Configuration} & \textbf{F1 (\%)} & \textbf{$\Delta$F1} \\
\midrule
\multicolumn{3}{l}{\textit{Filter Type}} \\
\quad No fusion (raw gyro) & 87.58 & --- \\
\quad Linear Kalman Filter & \textbf{91.10} & +3.52 \\
\quad Extended Kalman Filter & 90.12 & +2.54 \\
\midrule
\multicolumn{3}{l}{\textit{Process Noise ($Q_\phi$)}} \\
\quad 0.001 (very smooth) & 89.92 & -1.18 \\
\quad 0.005 (default) & \textbf{91.10} & --- \\
\quad 0.01 (responsive) & 90.21 & -0.89 \\
\bottomrule
\end{tabular}
\end{table}

%-------------------------------------------------------------------
\subsection{Normalization Strategy}
\label{app:normalization}

\begin{table}[H]
\caption{Normalization strategy comparison. ``Channel-aware'' applies z-score to accelerometer only, preserves radians for orientation.}
\label{tab:norm_ablation}
\centering
\begin{tabular}{lcc}
\toprule
\textbf{Normalization} & \textbf{F1 (\%)} & \textbf{$\Delta$F1} \\
\midrule
None & 88.23 & -2.87 \\
Unified z-score & 89.45 & -1.65 \\
\textbf{Channel-aware} & \textbf{91.10} & --- \\
\bottomrule
\end{tabular}
\end{table}

Channel-aware normalization preserves orientation angles in radians, maintaining their physical meaning ($\phi = 0$ represents upright position).

%-------------------------------------------------------------------
\subsection{Transformer Architecture}
\label{app:architecture}

\begin{table}[H]
\caption{Transformer architecture ablation. Default: 2 layers, 4 heads, 64 embed dim.}
\label{tab:arch_ablation}
\centering
\begin{tabular}{lccc}
\toprule
\textbf{Configuration} & \textbf{Params} & \textbf{F1 (\%)} & \textbf{$\Delta$F1} \\
\midrule
\multicolumn{4}{l}{\textit{Number of Layers}} \\
\quad 1 layer & 45K & 89.34 & -1.76 \\
\quad 2 layers (default) & 73K & \textbf{91.10} & --- \\
\quad 4 layers & 130K & 89.78 & -1.32 \\
\midrule
\multicolumn{4}{l}{\textit{Positional Encoding}} \\
\quad Sinusoidal & 73K & 90.12 & -0.98 \\
\quad \textbf{None} (default) & 73K & \textbf{91.10} & --- \\
\bottomrule
\end{tabular}
\end{table}

Positional encoding is not beneficial for fall detection. We hypothesize this occurs because: (1) fall signatures (impact spikes, orientation changes) are translation-invariant within windows---a fall occurring at the start or middle of a window has similar features; (2) our 4.3-second windows are short enough that long-range temporal dependencies are less critical than in sequence-to-sequence tasks; (3) the Conv1D projection already captures local temporal context (kernel size 8 $\approx$ 0.27s). This finding may not generalize to longer windows or non-impact activities where temporal position matters.

\textbf{SE+TAP Architecture Details.} The SE and TAP modules operate as follows:
\begin{itemize}
    \item \textbf{SE input:} Tensor of shape $(B, T, 64)$ after transformer encoder, where $B$ is batch size and $T=128$ timesteps
    \item \textbf{SE pooling:} Global average pooling over temporal dimension $\rightarrow$ $(B, 64)$
    \item \textbf{SE output:} Channel weights $(B, 64)$ broadcast to $(B, T, 64)$ via element-wise multiplication
    \item \textbf{TAP input:} SE-reweighted tensor $(B, T, 64)$
    \item \textbf{TAP output:} Attention-weighted temporal pooling $\rightarrow$ $(B, 64)$ for classification
\end{itemize}
This ordering (SE before TAP) allows channel reweighting to inform which features are emphasized during temporal attention.

%-------------------------------------------------------------------
\section{Experimental Configuration}
\label{app:config}

All experiments use the following configuration unless otherwise noted:

\textbf{Dataset \& Evaluation:}
\begin{itemize}
    \item SmartFallMM dataset, 21 young subjects (ages 18--35)
    \item 21-fold Leave-One-Subject-Out Cross-Validation
    \item Validation subjects: 48, 57 (held constant across folds)
\end{itemize}

\textbf{Preprocessing:}
\begin{itemize}
    \item Window: 128 frames ($\approx$4.3 seconds at 30 Hz)
    \item Stride: ADL=64 frames, Fall=16 frames (class-aware)
    \item Normalization: Channel-aware (z-score for acc only)
\end{itemize}

\textbf{Model Architecture:}
\begin{itemize}
    \item Transformer: 2 layers, 4 heads, embed\_dim=64
    \item Dual-stream: Conv1D projection with 32:32 split
    \item SE reduction ratio: 4
    \item Dropout: 0.5
\end{itemize}

\textbf{Training:}
\begin{itemize}
    \item Optimizer: AdamW, lr=$10^{-3}$, weight\_decay=$5\times10^{-4}$
    \item Batch size: 64, Epochs: 80
    \item Loss: Focal ($\alpha=0.75$, $\gamma=2.0$)
\end{itemize}

\textbf{Kalman Filter:}
\begin{itemize}
    \item Type: Linear Kalman Filter with adaptive measurement noise
    \item Process noise: $Q_\phi=0.005$, $Q_{\dot\phi}=0.01$
    \item Measurement noise: $R_\text{acc}=0.05$, $R_\text{gyro}=0.1$
    \item Adaptive threshold: $\tau=2.0g$, $R_\text{max}=10.0$
    \item Output: Euler angles $[\phi, \theta, \psi]$ in radians
\end{itemize}


